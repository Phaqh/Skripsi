\section{Jadwal Penelitian}

Penelitian ini direncanakan berlangsung selama enam bulan dengan
pembagian tahapan yang sistematis sesuai dengan alur metodologi
yang telah dijelaskan pada Bab 4. Penyusunan jadwal dilakukan
berdasarkan dependensi antar komponen sistem, dimulai dari
pembangunan lingkungan simulasi, implementasi model prediktif,
hingga evaluasi tertutup dan studi ablasi.

\subsection{Tahapan Penelitian}

Secara umum, tahapan penelitian dibagi menjadi sebelas fase utama
sebagai berikut:

\begin{enumerate}

\item \textbf{Perancangan dan Validasi Lingkungan IPD}

Tahap awal mencakup implementasi lingkungan
\textit{Iterated Prisoner's Dilemma} (IPD), termasuk:

\begin{itemize}
    \item Implementasi matriks payoff
    \item Implementasi fixed horizon ($T=100$)
    \item Implementasi geometric termination ($p_{term}=0.05$)
    \item Implementasi trembling-hand noise ($\epsilon \in \{0, 0.01, 0.05\}$)
    \item Validasi distribusi panjang episode
    \item Pengujian konsistensi reward dan terminasi
\end{itemize}

Tahap ini memastikan bahwa seluruh eksperimen dilakukan
pada lingkungan yang tervalidasi dan bebas dari kesalahan logika.

\item \textbf{Implementasi Tipe Lawan}

Meliputi pembangunan berbagai tipe lawan yang digunakan
dalam pengumpulan data dan evaluasi, yaitu:

\begin{itemize}
    \item Fixed mixed strategy
    \item Tit-for-Tat
    \item Win-Stay Lose-Shift
    \item Fictitious Play
    \item Q-Learning agent
\end{itemize}

Dilakukan pula validasi bahwa agen adaptif benar-benar
menunjukkan dinamika non-stasioner selama episode.

\item \textbf{Pengumpulan dan Validasi Dataset}

Tahap ini mencakup:

\begin{itemize}
    \item Simulasi 10.000 episode pelatihan
    \item Simulasi 2.000 episode validasi
    \item Simulasi 2.000 episode pengujian
    \item Verifikasi distribusi tipe lawan uniform
    \item Pemeriksaan data leakage
\end{itemize}

\item \textbf{Implementasi Model LSTM}

Meliputi:

\begin{itemize}
    \item Implementasi arsitektur LSTM satu layer (hidden size 64)
    \item Integrasi dropout untuk Monte Carlo dropout
    \item Implementasi scheduled sampling (0 → 0.3)
    \item Implementasi training pipeline (Adam, early stopping)
\end{itemize}

\item \textbf{Pelatihan Model Prediktif}

Meliputi:

\begin{itemize}
    \item Pelatihan model dengan teacher forcing
    \item Monitoring training dan validation loss
    \item Penyimpanan model terbaik berdasarkan validation loss
\end{itemize}

\item \textbf{Evaluasi Model Prediktif (Offline)}

Evaluasi one-step prediction dengan metrik:

\begin{itemize}
    \item Accuracy
    \item Cross-entropy loss
    \item Entropy prediktif
    \item Analisis pengaruh scheduled sampling
\end{itemize}

\item \textbf{Evaluasi Stabilitas Multi-Langkah (Open-loop)}

Meliputi:

\begin{itemize}
    \item Prediksi rekursif hingga horizon $k=10$
    \item Analisis pertumbuhan error terhadap horizon
    \item Analisis KL-divergence
    \item Analisis pertumbuhan variansi prediktif
\end{itemize}

\item \textbf{Implementasi Monte Carlo Rollout Planning}

Meliputi:

\begin{itemize}
    \item Implementasi estimator $\hat{Q}$
    \item Integrasi Monte Carlo dropout ($M=20$)
    \item Validasi kompleksitas $\mathcal{O}(N \cdot n \cdot k \cdot M)$
    \item Analisis konvergensi estimator terhadap $n$ dan $M$
\end{itemize}

\item \textbf{Evaluasi Closed-loop Agen}

Meliputi:

\begin{itemize}
    \item Simulasi episode penuh
    \item Evaluasi reward rata-rata
    \item Evaluasi win-rate
    \item Evaluasi tingkat kooperasi
    \item Uji robustnes terhadap noise dan horizon geometrik
\end{itemize}

\item \textbf{Studi Ablasi}

Meliputi:

\begin{itemize}
    \item Tanpa scheduled sampling
    \item Tanpa Monte Carlo dropout
    \item Horizon $k=1$
    \item Variasi jumlah kandidat aksi $N$
\end{itemize}

\item \textbf{Analisis Statistik dan Penulisan Laporan}

Tahap akhir mencakup:

\begin{itemize}
    \item Uji signifikansi statistik
    \item Analisis interval kepercayaan
    \item Visualisasi hasil eksperimen
    \item Penyusunan Bab 4 dan Bab 5 final
    \item Revisi berdasarkan masukan pembimbing
\end{itemize}

\end{enumerate}

\subsection{Rencana Waktu Pelaksanaan}

Rencana waktu pelaksanaan penelitian selama enam bulan
ditunjukkan pada Tabel berikut.

\begin{table}[H]
\centering
\caption{Rencana Jadwal Penelitian (2 Bulan / 8 Minggu)}
\begin{tabular}{|p{5.8cm}|c|c|c|c|c|c|c|c|}
\hline
\textbf{Kegiatan} & \textbf{M1} & \textbf{M2} & \textbf{M3} & \textbf{M4} & \textbf{M5} & \textbf{M6} & \textbf{M7} & \textbf{M8} \\
\hline
Perancangan lingkungan IPD & \checkmark & \checkmark & & & & & & \\
\hline
Implementasi tipe lawan & \checkmark & \checkmark & & & & & & \\
\hline
Pengumpulan dan validasi dataset & & \checkmark & \checkmark & & & & & \\
\hline
Implementasi model LSTM & & & \checkmark & \checkmark & & & & \\
\hline
Pelatihan model & & & & \checkmark & \checkmark & & & \\
\hline
Evaluasi model offline (one-step) & & & & & \checkmark & & & \\
\hline
Evaluasi open-loop multi-step & & & & & \checkmark & \checkmark & & \\
\hline
Implementasi Monte Carlo rollout & & & & & & \checkmark & \checkmark & \\
\hline
Evaluasi closed-loop agen & & & & & & & \checkmark & \\
\hline
Studi ablasi & & & & & & & \checkmark & \\
\hline
Analisis statistik dan visualisasi & & & & & & & & \checkmark \\
\hline
Penulisan dan revisi laporan & \checkmark & \checkmark & \checkmark & \checkmark & \checkmark & \checkmark & \checkmark & \checkmark \\
\hline
\end{tabular}
\end{table}

\subsection{Dependensi dan Risiko}

Beberapa dependensi penting dalam penelitian ini:

\begin{itemize}
    \item Implementasi rollout planning bergantung pada stabilitas model prediktif pada evaluasi open-loop.
    \item Evaluasi closed-loop tidak dapat dilakukan sebelum estimator nilai $\hat{Q}$ tervalidasi.
    \item Studi ablasi dilakukan setelah konfigurasi utama sistem stabil.
\end{itemize}

Risiko utama penelitian meliputi:

\begin{itemize}
    \item Akumulasi error prediksi pada horizon panjang
    \item Kompleksitas komputasi akibat $\mathcal{O}(N \cdot n \cdot k \cdot M)$
    \item Ketidakstabilan model pada lawan non-stasioner
\end{itemize}

Strategi mitigasi meliputi pengujian bertahap (offline → open-loop → closed-loop),
monitoring konvergensi estimator, serta pembatasan parameter eksperimen
untuk menjaga efisiensi komputasi.
