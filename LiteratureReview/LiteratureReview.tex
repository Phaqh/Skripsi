\documentclass[conference]{IEEEtran}
\IEEEoverridecommandlockouts

\usepackage{cite}
\usepackage{amsmath,amssymb,amsfonts}
\usepackage{algorithmic}
\usepackage{graphicx}
\usepackage{textcomp}
\usepackage{xcolor}

\def\BibTeX{{\rm B\kern-.05em{\sc i\kern-.025em b}\kern-.08em
    T\kern-.1667em\lower.7ex\hbox{E}\kern-.125emX}}

\begin{document}

\title{Your Paper Title Here}

\author{\IEEEauthorblockN{Author Name}
\IEEEauthorblockA{\textit{Department/Organization} \\
\textit{University/Company} \\
City, Country \\
email@example.com}
}

\maketitle

\begin{abstract}

\end{abstract}

\begin{IEEEkeywords}
keyword1, keyword2, keyword3 
\end{IEEEkeywords}

\section{Introduction}
Strategic interaction among autonomous agents—whether biological, social, or artificial—often unfolds over repeated encounters rather than isolated one-shot exchanges. The Iterated Prisoner’s Dilemma (IPD) has emerged as one of the most influential frameworks for understanding such long-term interactions, capturing the tension between short-term incentives to defect and long-term incentives to maintain cooperation. Since the seminal work of Axelrod, IPD has been used extensively to investigate cooperation, competition, trust, reciprocity, and adaptation in domains spanning evolutionary biology, social psychology, behavioral economics, and multi-agent systems. Its simplicity, well-defined payoff structure, and capacity to generate complex emergent behavior make IPD a canonical testbed for studying decision dynamics in repeated games.\cite{b1}

\subsection{Opponent Modelling}

\section{Related Work}
Your literature review and related work section.~\cite{b1}


\section{Methodology}
Description of your methodology.

\section{Results}
Your experimental results.

\section{Conclusion}
Your conclusions and future work.

\begin{thebibliography}{00}
\bibitem{b1} Author, ``Title,'' \textit{Journal}, Year.
\bibitem{b2} Author, ``Title,'' \textit{Journal}, Year.
\bibitem{b3} Author, ``Title,'' \textit{Journal}, Year.
\bibitem{b4} Author, ``Title,'' \textit{Journal}, Year.
\end{thebibliography}

\end{document}